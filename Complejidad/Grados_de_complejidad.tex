\documentclass[10pt,a4paper]{article}
\setlength{\parindent}{0pt}
\setlength{\parskip}{0.5em}
\usepackage[utf8]{inputenc}
\usepackage[spanish]{babel}
\usepackage{amsmath}
\usepackage{amsfonts}
\usepackage{amssymb}

\usepackage[colorlinks]{hyperref}
\hypersetup{colorlinks=true}
\hypersetup{urlcolor=blue}
\usepackage{cleveref}

\author{Juan Murua Olalde}
\title{Grados de complejidad}

\begin{document}

Los sistemas con los que tratamos pueden encuadrarse en uno de estos tres grados:
\begin{description}

\item[Sencillos]: Las influencias entre sus elementos son fijas, no varian. Su comportamiento pasado, presente y futuro se puede calcular y conocer con seguridad en cualquier ocasión.

\item[Complicados]: Las influencias entre sus elementos van variando según la situación de estos. Su comportamiento se puede estimar con un cierto margen de precisión para cualquier momento dado. Cuanto más lejos desde un momento de partida conocido, más cálculos para estimarlo. Cuanto más variabilidad en las influencias, mayor margen en la aproximación.
\\Estos sistemas se pueden subclasificar a su vez en:
\begin{description}
\item[Estables]: pequeñas variaciones, en cualquier parte, llevan siempre a pequeñas variaciones en el comportamiento.
\item[Inestables]: pequeñas variaciones en ciertas partes provocan grandes variaciones en el comportamiento.
\item[Caóticos]: pequeñas variaciones en ciertas partes provocan enormes variaciones en el comportamiento.
\end{description}

\item[Complejos]: Sistemas en los que sus elementos son tantísimos o la variabilidad de las influencias entre ellos es tal, que llegan a surgir ``comportamientos emergentes''. Comportamientos que, incluso para las personas más expertas, son ``sorprendentes'' y no hubieran sido posibles de prever antes de ser calculados o de ser observados.

\end{description}

A los humanos nos gusta tratar con sistemas del primer tipo, sencillos, con aquello que sabemos en todo momento cómo se va a comportar.

En la práctica, la mayoria de sistemas con los que interactuamos son del segundo tipo, complicados. Pero nos empeñamos (y muchas veces conseguimos) simplificarlos de tal forma que los podamos tratar como si fueran sencillos o, como mucho, complicados estables.

Durante la mayor parte de la historia de la humanidad, el único sistema del tercer tipo, complejo, con el que teniamos que lidiar realmente ha sido la propia sociedad humana o la metereologia. 

Se podria decir que, hasta hoy en día, todo aquello inestable, caótico o complejo era sistemáticamente ignorado. Pero ya no nos podemos permitir ese lujo. En las primeras décadas del siglo XXI, sobre todo cuando los sistemas informáticos andan por medio, está comenzando a ser posible (y necesario) tener en cuenta ese tipo de sistemas.

\hspace{1cm}

Algunos enlaces ilustrativos acerca de los sistemas complejos:
\\ \url{https://es.wikipedia.org/wiki/Sistema_complejo}
\\ \url{https://complexityexplained.github.io/}
\\ \url{https://isci.cl/}


\end{document}