\documentclass[spanish,10pt,a4paper,final,oneside]{article}
\setlength{\parindent}{0pt}
\setlength{\parskip}{0.5em}
\usepackage[spanish]{babel}
\usepackage[utf8]{inputenc}
\usepackage[a4paper, total={15cm, 25cm}]{geometry}

\addtolength{\skip\footins}{2pc plus 5pt}

\usepackage{longtable}
\setlength{\tabcolsep}{12pt}

\usepackage{amsmath}
\usepackage{amsfonts}
\usepackage{amssymb}

\usepackage{graphicx}
\graphicspath{ {./imagenes/} }

\usepackage[colorlinks]{hyperref}
\hypersetup{colorlinks=true}
\hypersetup{urlcolor=blue}
\usepackage{cleveref}

\usepackage{fancyhdr}
\fancyhf{}
%\fancyhead[RE]{\small\scshape\nouppercase{\leftmark}}
%\fancyhead[LO]{\small\scshape\nouppercase{\leftmark}}
\fancyhead[LE,RO]{\small\thepage}
\pagestyle{fancy}

\usepackage{authoraftertitle}
\title{Lista de herramientas para bricolage}
\author{Juan Murua Olalde}
\date{2014}

\begin{document}

\begin{center}\begin{LARGE}
\MyTitle
\end{LARGE}\end{center}
\begin{footnotesize}\rightline{inicio de redacción: \MyDate}
\rightline{últimos cambios: \today}\end{footnotesize}


\section*{Herramientas básicas:}
\begin{itemize}

\item Caja (o mejor bolsa) de herramientas robusta.

\item Algunas cajitas pequeñas, mejor con tapa.
\\para dejar en ellas temporalmente pequeñas piezas que vayamos desmontando, para que no se pierdan.

\item Destornillador Pozidrive PZ2 (el nuevo ``de estrella'').
\\(es un tipo de tirafondo o tornillo bastante habitual)

\item Destornillador Philips PH2 (el antiguo ``de estrella'').
\\(es un tipo de tirafondo o tornillo que era bastante habitual)

\item Destornillador plano (pala de unos 5, 6 o 7 mm de ancho).
\\(es un tipo de tirafondo o tornillo bastante habitual)

\item Destornillador buscapolos de electicista.
\\para clemas eléctricas pequeñas (suelen tener punta plana de 2 o 3 mm)

\item Destonillador Philips o Pozidrive pequeño.
\\para tapas de juguetes y mandos.
\\nota: los tamaños pequeños se suelen conocer como ``destornilladores de precisión'' o ``de relojero''

\item Tijeras de electricista.

\item Cutter robusto.

\item Juego de llaves fijas (en un rango desde 6-7 a 30-32).
\\para lidiar con tornillos/tuercas o con accesorios roscados de fontaneria

\item Hilo de teflón.
\\para sellar roscas de fontaneria

\item Juego de llaves Allen hexagonales o de bola .
\\para montaje de muebles

\item Alicate normal plano.

\item Alicate de puntas redondas.

\item Martillo de carpintero.
\\para lidiar con clavos u otros objetos que deseemos golpear (nota: si tenemos otro tipo de martillo, la función de la oreja saca-clavos puede hacerse también con unas tenazas)

\item Martillo con cabezas de plástico.
\\para ajustar/empujar suavemente objetos sobre los que no queramos dejar marca al golpearlos un poco

\item Maza de cantero (redonda, ``de campana'').
\\para golpear sobre cinceles, formones y similares

\item Cincel plano pequeño (de unos 5 a 10 mm de ancho) (también conocido como cincel ``de electricista'').
\\para trabajos de ``abrir huecos'' a golpes, donde el hueco no sea demasiado grande.

\item Formón (de unos 10 mm de ancho).
\\para hacer cajeras y otros huecos en madera

\item Sierra para madera.
\\nota: mejor una de corte fino; las sierras de corte basto cortan más rápido, pero dejan bordes demasiado irregulares que luego es necesario limar/lijar para dejarlos ``presentables''

\item Sierra para metal.
\\nota: sirve también para cortar plástico (por ejemplo, canaletas eléctricas o tubos de PVC) 

\item Limas de media caña: una basta y una fina.
\\nota: con lima de metales nos podemos apañar para todo tipo de materiales (metal, plástico, madera,\ldots); pero si vamos a usar mucho en madera, merece tener también limas específicas para ese material (sobre todo en el caso de la lima basta)

\item Tornillo de banco o de mordaza, para sujetar piezas.
\\nota: suele ser práctico si tiene una parte plana que se pueda usar como pequeño yunque

\item Cinta adhesiva `de montaje' de doble cara .
\\ para cuando no queramos hacer agujeros ni usar pegamento

\item Cinta americana.
\\ se corta muy fácil con la mano, es fuerte una vez pegada en su sitio, sirve para pegar o sujetar todo tipo de cosas,\ldots

\item Pegamento instantáneo (`Superglue' o similar).
\\$\rightarrow$ para pegar superficies que ajustan muy bien entre sí.
\\nota: superficies muy porosas (por ejemplo, cerámica) hay que humedecerlas con agua antes de aplicar el pegamento.
\\aviso: el cianocrilato pega muy rápido la piel humana y separar superficies pegadas grandes puede requerir hasta cirugía, tener mucho cuidado al sujetar las piezas con los dedos.

\item Pegamento epoxi de dos componentes (`Araldit' o similar).
\\$\rightarrow$ para pegar superficies irregulares con ligeros huecos (< 1 o 2 mm) entre sí.
\\modo de uso del epoxi: mezclar partes iguales de los dos componentes, mezclar bien y aplicar la mezcla a las superficies a pegar; con epoxi rápidos, en unos 5 minutos las partes se sujetan solas, con expoxi normales es necesario sujetarlas durante todo el curado; dejar curar durante 24 horas; el calor ayuda a conseguir uniones más resistentes.
\\consejo: las resinas epoxídricas ``pringan'' mucho, utilizar superficies de mezcla y espátulas desechables; procurar no tocar la resina con los dedos; el papel higiénico es un buen aliado para retirar restos de resina.

En lugar de pegamento epoxi, en algunas aplicaciones se puede utilizar lo se conoce como `cola de montaje'. Suele ser más sencilla de aplicar y mucho más limpia. Aunque, según en qué materiales, las uniones no suelen ser tan resistentes.

\item Masilla gris multiuso (de amasar como si fuera plastilina).
\\$\rightarrow$ para sujetar, rellenar huecos y reconstruir formas.

\item Pegamento de contacto (`Supergen' o similar).
\\$\rightarrow$ para zapatos, cuero y similares.
\\modo de uso: aplicar el pegamento en ambas superficies, apretarlas entre sí para extenderlo y separarlas, dejarlas al aire unos minutos hasta que el pegamento parezca seco, volver a apretar entre sí las superficies y sujetarlas así durante 24 horas.

\item Cola de carpintero blanca.
\\$\rightarrow$ para madera.
\\modo de uso: aplicar la cola en ambas superficies, sujetarlas bien apretadas entre sí, limpiar el exceso de cola de sobresalga (mismamente, con papel higiénico), dejar las piezas así durante 24 horas.

\item Cinta métrica (también conocida como flexómetro) de unos 5 o 10 m.
\\nota: para trabajos que requieran medir mucho, son muy útiles un par de reglas metálicas robustas (una de unos 40 cm y otra de unos 150 cm) y una escuadra de carpintero.
\\nota: para trabajos que requieran medir rápido, son muy útiles los medidores láser.

\item Algo para hacer marcas: lapiz, rotulador permanente, puntero de trazar en metal,\ldots

\item Taladro, que tenga un modo percutor (para agujeros en hormigón).
\item nota: Las herramientas eléctricas merece tenerlas en su correspondiente caja o maletín de transporte.

\item Granete automático autopercutor: punzón marcador de impacto (con muelle interno)
\\Al poderse usar con una sola mano, es muy práctico para marcar puntos de inicio de taladrado.

\item Juego de brocas ``universales'' para metal/plásticos/madera, de variados diámetros (el juego típico suele ir de 2 o 3 mm hasta 10 o 12 mm).
\\nota: si pensamos trabajar mucho en madera, interesa tener también brocas específicas para taladrar ese material

\item Broca de 6 mm para hormigón.

\item Tacos de 6 mm; Tirafondos de 4'5 o de 5 mm (x unos 40 mm longitud) (y otros más largos, según +grosor de lo que vayamos a sujetar) ; Arandelas variadas para esos tirafondos.

nota: las cosas quedan mejor sujetas con tirafondos de cabeza plana que con tirafondos normales (que tienen la cabeza avellanada), pero son más difíciles de encontrar.
\\nota: en paredes finas o en techos de pladur, madera o yeso, se necesitan tacos de mariposa o tacos expansivos.

\item Caja con compartimentos o con cajitas pequeñas en su interior.
\\para guardar tacos, tornillos o piezas pequeñas varias.

\item Atornillador a bateria.
\\nota: para un uso ocasional, son más cómodos los pequeños de mano; para trabajar mucho, son más robustos los de tipo taladro

\item Juego variado de puntas (bits): pozidrive (PZ), philips (PH), torkx (TX), planas, hexagonales,\ldots Con algún mango pequeño para usarlas manualmente (mejor si es de carraca). Y con algún extensor para acoplarlas al atornillador.
\\nota: suele ser conveniente tener aparte una punta (bit) Pozidrive PZ2 (o del tipo de cabeza que tengan los tirafondos o tornillos que usemos más a menudo), más o menos larga y que se pueda insertar directamente en el atornillador (sin extensor ni otro tipo de acoples)

\item Escalera pequeña, de dos o tres peldaños, que sea estable al subirnos y que sea fácil de mover de un sitio a otro;
\\para trabajos a pequeña altura

\item Cinturón porta-herramientas.

\item Linterna o frontal LED.

\item Gafas de protección.
\\nota: para trabajos largos, es más cómoda una pantalla protectora que unas gafas.

\item Guantes.
\\sobre todo si vamos a agarrar madera con bordes sin tratar (con astillas)

\end{itemize}


\section*{Otras herramientas interesantes:}

\begin{itemize}

\item Variados tipos de sargentos, o pinzas, o cintas con tensor de carraca, o\ldots
\\para sujetar piezas mientras las manipulamos o las pegamos

\item Nivel de burbuja de unos 30 cm.
\\para determinar líneas horizontales/verticales

\item Juego de destornilladores de precisión
\\para gafas, relojes,...

\item Pinzas para agarrar piezas pequeñas

\item Alicates de puntas redondas en ángulo
\\para manejar arandelas de sujeción de tornillos o bulones.

\item Sierra de vaivén (también conocida como sierra caladora o sierra de calar).
\\ hojas (bastas-rápidas-) para cortar madera ; hojas (finas-precisas-) para cortar madera ; hojas para cortar metales

\item Amoladora (tambien conocida como rotaflex)
\\ discos para cortar metal ; discos para cortar hormigón ; discos para cortar gres|cerámica ; discos para desgastar (esmeril)

\item Lijadora circular.
\\ discos lija de diversos granos (graduales, en varios pasos: de basto a fino) ; bayetas de pulir y pulimentos líquidos

\item nota: Todas las herramientas eléctricas merece tenerlas cada cual en su propia caja o maletín de transporte.

\item Gafas o pantalla de protección y Mascarilla antipolvo.

¡importante!: Trabajando con herramientas eléctricas, son indispensables las protecciones para ojos y pulmones. Además, a la larga, con ellas se trabaja más descansado.
\\nota: merece la pena tener protecciones de calidad (son más cómodas).

\item Caja de ingletar (para cortes manuales en ángulo)

\item Sierra circular con soporte ajustable a diversos ángulos.

\item Tijeras cortachapa.

\item Útil de remachar.
\\remaches de varios diámetros, según las boquillas del útil.

\item Llave(alicate) ajustable, ``de ?perro?'' , ``de ?pico de loro?'' 
\\para tubos y fontaneria. (¡ojo con ella!, agarra y permite girar cualquier elemento, pero siempre deja marcas y muy frecuentemente acabamos rompiendo lo que tratamos de manipular)

\item Juego de llaves de tubo
\\para manejar tornillos/tuercas en espacios donde no entran las llaves fijas.

\item Alicates corta alambres ; Tenazas corta varillas
\\(más o menos grandes, según los diámetros a cortar)

\item Maza ; Piqueta
\\(más o menos grandes, según lo que deseemos demoler)

\item Palanqueta
\\(para evitar tentaciones de hacer palanca con otras herramientas y, así, evitar deformarlas)


\end{itemize}

\section*{Para trabajos sencillos de gremios varios:}

\begin{itemize}

\item Pintura: cinta de carrocero (para proteger lo que no deseamos pintar/emplastar), cartón corrugado (para el suelo), manta de plástico/lona/tela (para cubrir objetos difíciles de mover), cepillo de puas metálicas con mango largo (rascar), espátula (rascar), cepillo suave (limpiar, retirar polvo), masilla|yeso (`Aquaplast' o similar), juego de espátulas de carrocero, bloque para lijar, papel de lija basto, papel de lija fino, cubeta para pintura con escurridor, rodillo, extensor para rodillo, bandeja pequeña para pintura, brocha plana (de unos 50 mm), destornillador plano grande y largo (para abrir botes y para remover pintura),\ldots 

\item Albañileria: capazos de goma (para mezclas, para transportar escombro,...), pala (la hoja con punta es más cómoda para hincar en montones de escombro o de arena) (la hoja recta es más cómoda para recoger), paleta de albañil plana, paletilla de albañil triangular (con punta redondeada), llana dentada (de unos 30-40 cm), llana sin dientes (de unos 30-40 cm), llana con goma/esponja, nivel de burbuja corto (de unos 30 o 40 cm), nivel de burbuja largo (de 1 m o mayor), plomada, hilo y polvo de marcar, clavos de acero, escuadra, reglas de albañil (una de 1 m y otra de unos 2 m), cortadora de azulejos, alicate ``de morder'' azulejos, corona de perforación de unos 44 mm (agujeros para tubos) y de unos 68 mm (agujeros para mecanismos eléctricos), caballetes con tablones o andamio corrido (para trabajar cómodamente en altura a lo largo de una pared),\ldots

\item Fontaneria: bandejas y cubos para recoger liquido, fregona muy absorbente, trapos para empapar y hacer barreras, desatascador de goma, cable desatascador (es como una espiral metálica con un mango manivela), bomba desatascadora (es como un inflador), pasacables de electricista grueso de unos 20 o 30 m de largo, hilo de teflón, llave fija grande de 36,\ldots

Para tubos de plástico PVC: sierra de metales o cortador de tubos, pegamento específico para PVC. Para tubos de cobre: cortador de tubos, rebañador de bordes, soplete de gas, decapante, hilo de soldadura blanda. Para tubos de acero: sierra de metales o cortador de tubos, rebañador de bordes, machos y terrajas de roscar, aceite de roscar.

\item Electricidad: comprobador electrico o multímetro de electricista o pinza amperimétrica (como mínimo, algo para medir continuidad/voltaje; interesante medir amperaje/potencia; resto de medidas son profesionales), tijeras de electricista, destornillador de electricista (cruz-plano), pasacables de varias longitudes (uno corto de unos 3 m. y uno más largo de unos 10 m.), cinta aislante (para sujetar cables al pasacables o entre ellos mientras los deslizamos), clemas de varios tamaños: 6 mm, 10 mm, 16 mm,\ldots (para empalmar/conectar cables entre sí), segueta para cortar canaletas de plástico, cutter robusto, escalera pequeña (de dos o tres peldaños) estable y fácil de mover (para llegar cómodamente al trabajar en registros altos y techos) (para sentarse al trabajar en interruptores y enchufes a media altura ;-), bridas cortas para sujetar cables, bridas largas para sujetar mazos de cables, maceta y cincel de electricista (para hacer huecos donde empotrar cajas), yeso o cemento rápido para fijar las cajas empotradas (¡ojo!, endurece en pocos minutos) (nota: rellenar las cajas con papel para que el cemento no penetre en ellas), tupperware rectangular (donde mezclar pequeñas cantidades de yeso/cemento), espátula (para mezclar yeso/cemento), espátula de carrocero (para aplicar y alisar yeso/cemento),\ldots

\end{itemize}

\end{document}
