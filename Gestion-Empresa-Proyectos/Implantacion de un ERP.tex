\documentclass[spanish,12pt,a4paper,final,oneside]{article}
\setlength{\parindent}{0pt}
\setlength{\parskip}{0.5em}
\usepackage[spanish]{babel}
\usepackage[utf8]{inputenc}
\usepackage[a4paper, total={15cm, 23cm}]{geometry}

\addtolength{\skip\footins}{2pc plus 5pt}

\usepackage{enumitem}
\setlist{topsep=0pt}

\usepackage{longtable}
\setlength{\tabcolsep}{12pt}

\usepackage{amsmath}
\usepackage{amsfonts}
\usepackage{amssymb}

\usepackage{graphicx}
\graphicspath{ {./imagenes/} }

\usepackage[colorlinks]{hyperref}
\hypersetup{colorlinks=true}
\hypersetup{urlcolor=blue}
\usepackage{cleveref}

\usepackage{fancyhdr}
\fancyhf{}
%\fancyhead[RE]{\small\scshape\nouppercase{\leftmark}}
%\fancyhead[LO]{\small\scshape\nouppercase{\leftmark}}
\fancyhead[LE,RO]{\small\thepage}
\pagestyle{fancy}

\usepackage{authoraftertitle}
\title{Implantación de un ERP}
\author{Juan Murua Olalde}
\date{27/04/2022}

\begin{document}

\begin{center}\begin{LARGE}
\MyTitle
\end{LARGE}\end{center}
\begin{footnotesize}\rightline{inicio de redacción: \MyDate}
\rightline{últimos cambios: \today}\end{footnotesize}

\hypersetup{linkcolor=black}
%\tableofcontents

\vspace{1cm}


\section{Selección}
Es importante involucrar desde el principio a una parte amplia y variada de las personas que van a verse afectadas. Por ejemplo, en la lista de requisitos y en las formaciones/demostraciones han de participar todos los departamentos/delegaciones/filiales/\ldots

Es importante recoger el sentir y la opinión generales.
\\Es importante que todas las personas se sientan informadas de lo que se va a hacer.
\\Es importante que quede claro que el nuevo ERP es importante\footnote{ Hacer ver que Dirección está realmente implicada. Es decir, que está convencida del valor del ERP y que pone los medios que sean necesarios, a todos los niveles.}. 


\subsection{Lista de requisitos funcionales operativos}
Se trata de recoger qué hacemos, y qué datos empleamos para hacerlo, en los procesos de nuestro día a día.

Es un punto de partida para comenzar a hablar con los partners de los distintos ERPs acerca de cómo cubren sus soluciones nuestros requerimientos.

\subsection{Formación y demostraciones}
Se trata de ver cómo cubren algunos de nuestros requisitos (los más importantes) todos y cada uno de los candidatos a nuevo ERP.

Con la doble intención de:
\begin{itemize}

\item Tener una impresión de cómo lo hacen.

\item Calibrar el grado de entendimiento:
\begin{itemize}
\item que tiene el partner de nuestro negocio,
\item que tenemos nosotros de la solución propuesta.
\end{itemize}

\end{itemize}

\subsection{Evaluar el ecosistema de partners}
Se trata de ver los posibles partners que podriamos tener en los distintos paises donde estemos implantados.

Hablando con ellos para:
\begin{itemize}
\item evaluar el soporte que podemos recibir,
\item evaluar el grado de colaboración al que podemos aspirar.
\end{itemize}


\section{Implantación}

Es importante planificar entregables concretos y tangibles a lo largo del proyecto. Se ha de trabajar de forma iterativa, implementando algo ``que se pueda ver'' cada pocos meses.
\\Huir de proyectos ``monolíticos'' cerrados, ``llave en mano'', proyectos de entregar ``todo o nada''.

(*) Obviamente, también se han de planificar los correspondientes pagos a lo largo del proyecto. Aunque de eso ya se encargará el partner\ldots ;-) (pocos están dispuestos a asumir un solo pago ``todo o nada'' al final).


\subsection{Seguimiento}
Es necesario llevar al día (algunos documentos son vivos, su información va variando) y conservar (almacenamiento con gestión de versiones):
\begin{itemize}

\item \textbf{Ofertas} y \textbf{contratos} firmados.

\item \textbf{Documentos de requisitos} (casos de uso) y \textbf{resultados de los test} que verifican que se cumplen, tanto unitarios (con casos estándares teóricos) como funcionales (con casos reales nuestros)

\item \textbf{Bitácora} del proyecto, recogiendo el ``timeline'' de lo que ha ido sucediendo. Documentando bien todas las decisiones importantes, en el mismo momento en que se están evaluando y en mismo momento en que toman.

\item Copias de seguridad del software desarrollado, ``snapshots'' de las distintas fases del proyecto. De tal forma que se puedan volver a montar los entornos tal y como estaban en un punto determinado del tiempo, cara a demostrar/evaluar/refutar afirmaciones futuras.

\item \textbf{Resultados obtenidos}, recogiendo las partes que se han utilizado (con tiempo/volumen de utilización) y las partes que no se han podido utilizar (con explicación de por qué). Servirá para elaborar un informe de valor entregado vs. precio pagado vs. coste incurrido.

\end{itemize}


\subsection{Arranque}
Todas las personas que vayan a utilizar los nuevos sistemas han tenido que recibir formación previa. Por lo menos la suficiente para que ``les suene'' lo que van a utilizar y sepan buscar ayuda cuando se atasquen.

Contar con que seguirán necesitando formación y soporte después del arranque, hasta que ``se sientan cómodas'' utilizando el nuevo ERP y los nuevos procesos.

(*) Obviamente, ninguna formación puede ser efectiva sin motivación ni ganas de aprender.



\subsection{Post-arranque}
La fase de estabilización es clave.

Tanto nosotros como el partner hemos de seguir colaborando y poniendo los recursos que sean necesarios. Hasta que el software ``entre en velocidad de crucero'' en su uso habitual.

Durante el tiempo que sea necesario; hasta que el número de incidencias sea razonable.

(*) Todo software tiene errores. Pero lo que no se puede admitir es que tenga errores bloqueantes (aquellos que impiden trabajar en procesos clave para el normal funcionamiento de la empresas y que no son resolubles en un periodo corto de tiempo).


\end{document}